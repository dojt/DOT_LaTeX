% defs.common.tex

\newcommand{\from}{\leftarrow}
\newcommand{\downto}{\searrow}
\newcommand{\upto}{\nearrow}
\newcommand{\aand}{\mathrel\&}

\newcommand{\Frowney}{${{}_{.\,.}}\above0pt{}^{\frown}$}
\newcommand{\Smiley}{{\ensuremath{{{}_{.\,.}}\above0pt{}^{\smile}}}}

\newlength{\hspaceforlengthglumpf}
\newcommand{\hsfor}[1]{\settowidth{\hspaceforlengthglumpf}{#1}\hspace{\hspaceforlengthglumpf}}
\newcommand{\hsforminus}[1]{\settowidth{\hspaceforlengthglumpf}{#1}\hspace{-\hspaceforlengthglumpf}}

\newcommand{\onespace}{\mspace{1mu}}


\newcommand{\stepnr}[1]{{\footnotesize #1}}

\newcommand{\stbox}[2]{\text{\tiny\parbox{#1}{\centering\baselineskip=0.5 \baselineskip #2}}}

\newcommand{\ubtxt}[3]{\underbrace{#1}_{\stbox{#2}{#3}}}
\newcommand{\obtxt}[3]{\overbrace{#1}^{\stbox{#2}{#3}}}
\newcommand{\utxt}[3]{\mathop{#1}\limits_{\stbox{#2}{#3}}}

\renewcommand{\em}{\sl}

\newcommand{\cmmt}[1]{\text{\footnotesize[#1]}}

% continuous numberin in enumis
\newcommand{\prepareEnumiSave}[1]{\newcounter{myenumiaidsyfgusiwuebfoauishdf#1}}
\newcommand{\saveEnumi}[1]{\setcounter{myenumiaidsyfgusiwuebfoauishdf#1}{\value{enumi}}}
\newcommand{\restoreEnumi}[1]{\setcounter{enumi}{\value{myenumiaidsyfgusiwuebfoauishdf#1}}}

%\newcommand{\case}[1]{\par\smallskip\noindent\textit{#1}}

%%%%%%%%%%%%%%%%%%%%%%%%%%%%%
%% math relations & Operators

\newcommand{\approxcmmt}[1]{\mathrel{\mathop\approx\limits_{\stbox{1.5cm}{#1}}}}
\newcommand{\eqcmmt}[1]{\mathrel{\mathop=\limits_{\stbox{1.5cm}{#1}}}}
\newcommand{\lecmmt}[1]{\mathrel{\mathop\le\limits_{\stbox{1.5cm}{#1}}}}
\newcommand{\gecmmt}[1]{\mathrel{\mathop\ge\limits_{\stbox{1.5cm}{#1}}}}
\newcommand{\eqcomment}[2]{\mathrel{\mathop=\limits_{\stbox{#1}{#2}}}}
\newcommand{\eqshrtcmmt}[1]{\mathrel{\mathop=\limits_{#1}}}

\newcommand{\eqcmt}[1]{\mathrel{\mathop=\limits_{#1}}}
\newcommand{\lecmt}[1]{\mathrel{\mathop\le\limits_{#1}}}
\newcommand{\gecmt}[1]{\mathrel{\mathop\ge\limits_{#1}}}
\newcommand{\lesssimcmt}[1]{\mathrel{\mathop\lesssim\limits_{#1}}}
\newcommand{\gtrsimcmt}[1]{\mathrel{\mathop\gtrsim\limits_{#1}}}

\newcommand{\musteq}{\stackrel{!}{=}}
\newcommand{\mustleq}{\stackrel{!}{\le}}
\newcommand{\questioneq}{\stackrel{?}{=}}


\newcommand{\then}{\mathrel{\mathop{\;\Rightarrow\;}}}
\newcommand{\into}{\hookrightarrow}
\newcommand{\onto}{\twoheadrightarrow}
\newcommand{\leply}{\mathrel{\le_p}}

\DeclareMathOperator{\id}{id}
\DeclareMathOperator{\ord}{ord}
\DeclareMathOperator{\dist}{dist}
\DeclareMathOperator{\diam}{diam}
\DeclareMathOperator{\divg}{div}
\DeclareMathOperator{\rot}{rot}
\DeclareMathOperator{\defect}{def}
\DeclareMathOperator{\codim}{codim}
\DeclareMathOperator{\sgn}{sgn}
\DeclareMathOperator{\vrt}{vert}
\DeclareMathOperator{\argmax}{argmax}
\DeclareMathOperator{\argmin}{argmin}
\DeclareMathOperator{\tr}{tr}
\DeclareMathOperator{\polylog}{polylog}
\DeclareMathOperator{\lnln}{lnln}
\DeclareMathOperator{\res}{res}
\DeclareMathOperator{\rk}{rk}
\DeclareMathOperator{\vol}{vol}

\renewcommand{\div}{\divg}
\newcommand{\Id}{\mathrm{Id}}

%% Matricess / Operators
\DeclareMathOperator{\Diag}{Diag}

% Sets
\DeclareMathOperator{\Supp}{Supp}
\DeclareMathOperator{\Spec}{Spec}
\DeclareMathOperator{\Relint}{Relint}
\DeclareMathOperator{\Img}{Img}
\DeclareMathOperator{\Ker}{Ker}

\DeclareMathOperator{\Lin}{Lin}
\DeclareMathOperator{\Spn}{Span}
\DeclareMathOperator{\Aff}{Aff}
\DeclareMathOperator{\Cvxhull}{Cvxhull}
\DeclareMathOperator{\Cvxcone}{Cvxcone}

%\DeclareMathOperator{\Real}{Re}
%\DeclareMathOperator{\Imag}{Im}

\newcommand{\orth}{\bot}

\newcommand{\One}{\mathbf{1}}
\newcommand{\Zero}{\mathbf{0}}
%\newcommand{\place}{{\text{\footnotesize$\oblong$}}}

\newcommand{\ONE}{\mathbb{1}}

\newcommand{\lt}{\left}
\newcommand{\rt}{\right}
\newcommand{\lft}{\bigl}
\newcommand{\rgt}{\bigr}
\newcommand{\Lft}{\Bigl}
\newcommand{\Rgt}{\Bigr}

\newcommand{\T}{{\mspace{-1mu}\scriptscriptstyle\top\mspace{-1mu}}}
\newcommand{\polar}{\vartriangle}
\newcommand{\poOps}{\Diamond}
\newcommand{\blOps}{\sharp}

\newcommand{\lmit}{\;\;\vrule\;\;}
\newcommand{\Mit}{\;\;\vrule\;\;}


\newcommand{\mIt}[1]{{\mathit{#1}}}

% \newcommand{\Mod}{\!\big/\!}
% \newcommand{\MOd}{\!\Big/\!}

%%\newcommand{\cupdj}{\stackrel{.}{\cup}}
%\DeclareMathOperator*{\bigcupdj}{{\bigcup\kern-1.7ex {}^.}\,}
%\newlength{\DOTwidthofabigcup}
%\DeclareMathOperator*{\bigcupdj}{{\settowidth{\DOTwidthofabigcup}{\bigcup}{\bigcup\kern-\DOTwidthofabigcup/2{}^.}}}
%\newcommand{\bigcupdj}{{\bigcup^.}\limits}

\newcommand{\dotcup}{\mathbin{\dot\cup}}

\renewcommand{\Hat}{\wedge}

\newcommand{\comma}{\text{, }}
\newcommand{\Order}{O}
\newcommand{\OrderEq}{\ensuremath{\Theta}}

\newcommand{\modu}[2]{%
{{}^{\textstyle #1}\!\!\Big/\!\!{}_{\textstyle #2}}}
\newcommand{\restr}[2]{\lt.{#1}\vrule{}_{#2}\rt.}
\newcommand{\vrstr}[2]{\restr{#1}{#2}}
\newcommand{\sstack}[1]{{\substack{#1}}}
\newcommand{\sstck}[1]{{\substack{#1}}}

\newcommand{\close}[1]{\overline{#1}}
\newcommand{\cj}[1]{\overline{#1}}
%\newcommand{\widebar}[1]{\overline{#1}}
\newcommand{\widevec}[1]{\overrightarrow{#1}}

\DeclareMathOperator{\pr}{pr}

\newcommand{\dann}{\Rightarrow}
\newcommand{\pii}{{\pi i}}

\newcommand{\edzpii}{{\frac{1}{2\pii}}}
\newcommand{\nfrac}[2]{{\nicefrac{#1}{#2}}}

\newcommand{\cplmt}{\complement}

\newcommand{\symdiff}{{\vartriangle}}

\newcommand{\AAa}{\mathbb{A}}
\newcommand{\BB}{\mathbb{B}}
\newcommand{\CC}{\mathbb{C}}
\newcommand{\DD}{\mathbb{D}}
\newcommand{\EE}{\mathbb{E}}
\newcommand{\FF}{\mathbb{F}}
\newcommand{\HH}{\mathbb{H}}
\newcommand{\KK}{\mathbb{K}}
\newcommand{\LL}{\mathbb{L}}
\newcommand{\MM}{\mathbb{M}}
\newcommand{\NN}{\mathbb{N}}
\newcommand{\PP}{\mathbb{P}}
\newcommand{\QQ}{\mathbb{Q}}
\newcommand{\RR}{\mathbb{R}}
\newcommand{\Ss}{\mathbb{S}}
\newcommand{\TT}{\mathbb{T}}
\newcommand{\UU}{\mathbb{U}}
\newcommand{\ZZ}{\mathbb{Z}}
\newcommand{\kk}{\mathbb{k}}

\newcommand{\CCpkt}{\C^{\bullet}}

\newcommand{\SPH}{\mathbb{S}}
\newcommand{\PR}{\mathbb{PR}}
\newcommand{\PC}{\mathbb{PC}}

\newcommand{\DDelta}{\mathbb{\Delta}}
\newcommand{\Spx}{\mathbb{\Delta}}

%% E x p e c t a t i o n

\DeclareMathOperator{\bbbExp}{\EE}
\newcommand{\bbblExp}{\bbbExp\displaylimits}
\DeclareMathOperator{\bbbEXp}{\text{\Large$\EE$}}
\newcommand{\bbblEXpo}{\bbbEXp\displaylimits}
\newcommand{\bbblEXp}{\bbbEXp\displaylimits}
\DeclareMathOperator{\bbbEXP}{\text{\huge$\EE$}}
\newcommand{\bbblEXP}{\bbbEXP\displaylimits}

\DeclareMathOperator{\expct}{\mathbf E}
\DeclareMathOperator{\Expct}{\text{\large$\mathbf E$}}

\DeclareMathOperator*{\Prb}{\mathbb{P}}
\DeclareMathOperator*{\Exp}{\mathbb{E}}
\DeclareMathOperator*{\HExp}{\text{\huge$\mathbb{E}$}}
\DeclareMathOperator*{\LExp}{\text{\Large$\mathbb{E}$}}
\DeclareMathOperator*{\lExp}{\text{\large$\mathbb{E}$}}
\DeclareMathOperator*{\Var}{\mathbb{V}\mspace{-2mu}ar}
\DeclareMathOperator*{\Covar}{\mathbb{Covar}}
\DeclareMathOperator{\IndicatorOp}{\mathbf{1}}
\newcommand{\Ind}{\IndicatorOp}
\newcommand{\Cov}{\Covar}
\newcommand{\I}[1]{\One_{#1}}

%%%%%%%%%%%%%%%%%%%%%%%%%%%%%%%%%%

\newcommand{\Bd}{\mathcal B}

\newcommand{\fB}{{ \mathfrak{B} }}
\newcommand{\fA}{{ \mathfrak{A} }}
\newcommand{\fT}{{ \mathfrak{T} }}
\newcommand{\fS}{{ \mathfrak{S} }}
\newcommand{\fE}{{ \mathfrak{E} }}

\newcommand{\yps}{\upsilon}
\newcommand{\Yps}{\Upsilon}
\newcommand{\eps}{\varepsilon}

\newcommand{\nMtx}[1]{{\begin{smallmatrix}#1\end{smallmatrix}}}
\newcommand{\Mtx}[1]{{\bigl(\begin{smallmatrix}#1\end{smallmatrix}\bigr)}}
\newcommand{\Mtxv}[1]{{\lt(\begin{smallmatrix}#1\end{smallmatrix}\rt)}}
\newcommand{\bMtx}[1]{{\begin{pmatrix}#1\end{pmatrix}}}

\newcommand{\mtt}[1]{{\text{\tt #1}}}
\newcommand{\ID}{\Mtx{1&0\\0&1}}
\newcommand{\bID}{\bMtx{1&0\\0&1}}
\newcommand{\Mabcd}{\Mtx{a&b\\c&d}}
\newcommand{\bMabcd}{\bMtx{a&b\\c&d}}

\newcommand{\scp}[2]{{\lt.\lt( #1 \,\right| #2 \rt)}}
\newcommand{\pNm}[2]{{\lt\lVert #1 \rt\rVert_{#2}}}

\newcommand{\floor}[1]{\lfloor{#1}\rfloor}
\newcommand{\ceil}[1]{\lceil{#1}\rceil}
\newcommand{\Ceil}[1]{\lt\lceil{#1}\rt\rceil}
\newcommand{\Floor}[1]{\lt\lfloor{#1}\rt\rfloor}

\newcommand{\abs}[1]{{\lvert{#1}\rvert}}                        \newcommand{\Nm}[1]{{\lVert{#1}\rVert}}
\newcommand{\Abs}[1]{{\lt\lvert{#1}\rt\rvert}}                  \newcommand{\NM}[1]{{\lt\lVert{#1}\rt\rVert}}
\newcommand{\absb}[1]{{\bigl\lvert{#1}\bigr\rvert}}             \newcommand{\Nmb}[1]{{\bigl\lVert{#1}\bigr\rVert}}
\newcommand{\absB}[1]{{\Bigl\lvert{#1}\Bigr\rvert}}             \newcommand{\NmB}[1]{{\Bigl\lVert{#1}\Bigr\rVert}}
\newcommand{\absbg}[1]{{\biggl\lvert{#1}\biggr\rvert}}          \newcommand{\Nmbg}[1]{{\biggl\lVert{#1}\bigbr\rVert}}
\newcommand{\absBg}[1]{{\Biggl\lvert{#1}\Biggr\rvert}}          \newcommand{\NmBg}[1]{{\Biggl\lVert{#1}\Bigbr\rVert}}

\newcommand{\cip}[2]{({#1}\mid{#2})}
\newcommand{\cipb}[2]{\big({#1}\bigm|{#2}\big)}
\newcommand{\cipB}[2]{\Big({#1}\Bigm|{#2}\Big)}
\newcommand{\cipbg}[2]{\bigg({#1}\biggm|{#2}\bigg)}
\newcommand{\cipBg}[2]{\Bigg({#1}\Biggm|{#2}\Bigg)}

\newcommand{\rip}[2]{\langle{#1} \,,\, {#2}\rangle}
\newcommand{\ripb}[2]{\big\langle{#1} \,,\, {#2}\big\rangle}
\newcommand{\ripB}[2]{\Big\langle{#1} \,,\, {#2}\Big\rangle}
\newcommand{\ripbg}[2]{\bigg\langle{#1}\;,\;{#2}\bigg\rangle}
\newcommand{\ripBg}[2]{\Bigg\langle{#1}\;,\;{#2}\Bigg\rangle}

\newcommand{\dual}[2]{\langle{#1}:{#2}\rangle}
\newcommand{\dualb}[2]{\big\langle{#1}:{#2}\big\rangle}
\newcommand{\dualB}[2]{\Big\langle{#1}:{#2}\Big\rangle}
\newcommand{\dualbg}[2]{\bigg\langle{#1}\;:\;{#2}\bigg\rangle}
\newcommand{\dualBg}[2]{\Bigg\langle{#1}\;:\;{#2}\Bigg\rangle}

\newcommand{\braket}[2]{{\lt< #1 \mid #2 \rt>}}
\newcommand{\ketbra}[2]{{\lvert #1 \rangle\langle #2 \rvert}}
\newcommand{\bra}[1]{{\langle #1 \rvert}}
\newcommand{\ket}[1]{{\lvert  #1 \rangle}}
\newcommand{\BRA}[1]{{\lt\langle #1 \rt\rvert}}
\newcommand{\KET}[1]{{\lt\lvert  #1 \rt\rangle}}
\newcommand{\brab}[1]{{\bigl\langle #1 \bigr\rvert}}
\newcommand{\ketb}[1]{{\bigl\lvert  #1 \bigr\rangle}}
\newcommand{\braB}[1]{{\Bigl\langle #1 \Bigr\rvert}}
\newcommand{\ketB}[1]{{\Bigl\lvert  #1 \Bigr\rangle}}
\newcommand{\brabg}[1]{{\biggl\langle #1 \biggr\rvert}}
\newcommand{\ketbg}[1]{{\biggl\lvert  #1 \biggr\rangle}}
\newcommand{\braBg}[1]{{\Biggl\langle #1 \Biggr\rvert}}
\newcommand{\ketBg}[1]{{\Biggl\lvert  #1 \Biggr\rangle}}
\newcommand{\rsp}[2]{\lt<#1\,,\, #2 \rt>}

\newcommand{\iprod}{\bullet}
\newcommand{\ip}[1]{\lt(#1\rt)}
\newcommand{\ipn}[1]{(#1)}
\newcommand{\ipb}[1]{\bigl(#1\bigr)}
\newcommand{\ipB}[1]{\Bigl(#1\Bigr)}
\newcommand{\ipbg}[1]{\biggl(#1\biggr)}
\newcommand{\ipBg}[1]{\Biggl(#1\Biggr)}



%\definecolor{purp}{named}{Purple}
\newcommand{\xyATabular}[5]{
\vspace{0.1cm}
\begin{tabular}{l@{}l}
   & \fcolorbox{black}{#1}{\parbox[c][0.3cm]{1.1cm}{\centering #2}} \\
\fcolorbox{black}{#3}{\parbox[c][1.1cm]{0.3cm}{#4}} &
        \fbox{\parbox[c][1.1cm]{1.1cm}{\centering #5}}
\end{tabular}
\vspace{0.1cm}
}
\newcommand{\rlstopline}[9]{
\begin{picture}(#1,#3)(0,-#2)
        \put(0,0){\line(1,0){#1}}
        \put(0,0){\line(0,1){#2}}
        \put(0,0){\line(0,-1){#2}}
        \put(#1,0){\line(0,1){#2}}
        \put(#1,0){\line(0,-1){#2}}
        \put(#4,#5){#6}
        \put(#7,#8){#9}
\end{picture}
}
\newcommand{\tridipicture}[2]{
\unitlength=#1
\begin{picture}(5,5)(0,0)
\put(0,5){\line(1,-1){5}}
\put(1,5){\line(1,-1){4}}
\put(0,4){\line(1,-1){4}}
#2
\end{picture}
}

\newcommand{\Kone}{\ding{"0C0}}
\newcommand{\Ktwo}{\ding{"0C1}}
\newcommand{\Kthree}{\ding{"0C2}}
\newcommand{\Kfour}{\ding{"0C3}}
\newcommand{\Kfive}{\ding{"04C}}
\newcommand{\Ksix}{\ding{"05C}}
\newcommand{\Kseven}{\ding{"60C}}
\newcommand{\Keight}{\ding{"0C7}}
\newcommand{\Knine}{\ding{"0C8}}
\newcommand{\Kzero}{\ding{"0CA}}


% Damit nach Abkuerzungen nicht derselbe Abstand wie nach einer
% Satzendeinterpunktion erfolgt, wird xspace eingesetzt.
% Die verwendeten Abkuerzungen werden hier der Einfachheit halber
% definiert:
%\newcommand{\Nr}{Nr.}

\newlength{\algotabbingwidth}
\setlength{\algotabbingwidth}{1cm}
\newenvironment{algo}{%
  \begin{tabbing}
    \hspace*{\algotabbingwidth}\=%
    \hspace{\algotabbingwidth}\=%
    \hspace{\algotabbingwidth}\=%
    \hspace{\algotabbingwidth}\=%
    \hspace{\algotabbingwidth}\=%
    \hspace{\algotabbingwidth}\=%
    \hspace{\algotabbingwidth}\=%
    \hspace{\algotabbingwidth}\=%
    \hspace{\algotabbingwidth}\=%
    \hspace{\algotabbingwidth}\=%
    \hspace{\algotabbingwidth}\=%
    \hspace{\algotabbingwidth}\=%
    \hspace{\algotabbingwidth}\=%
    \kill\\
    }{%
  \end{tabbing}%
}
\newenvironment{algoShort}{%
  \begin{tabbing}
    \hspace*{\algotabbingwidth}\=%
    \hspace{\algotabbingwidth}\=%
    \hspace{\algotabbingwidth}\=%
    \hspace{\algotabbingwidth}\=%
    \hspace{\algotabbingwidth}\=%
    \hspace{\algotabbingwidth}\=%
    \hspace{\algotabbingwidth}\=%
    \hspace{\algotabbingwidth}\=%
    \hspace{\algotabbingwidth}\=%
    \hspace{\algotabbingwidth}\=%
    \hspace{\algotabbingwidth}\=%
    \hspace{\algotabbingwidth}\=%
    \hspace{\algotabbingwidth}\=%
    \kill
    }{%
  \end{tabbing}%
}
%% \newlength{\algoboxwidth}%
%% \newlength{\algolabelwidth}%
%% \newcommand{\algbox}[3]{%
%%   \settowidth{\algolabelwidth}{#1~}%
%%   \setlength{\algoboxwidth}{-\algolabelwidth}%
%%   \addtolength{\algoboxwidth}{-#2cm}%
%%   \addtolength{\algoboxwidth}{\textwidth}%
%%   #1~\parbox[t]{\algoboxwidth}{\raggedright #3\vspace{1mm}
%%     }%
%% }

\newcommand{\STOP}[0]{{STOP}}

\newcommand{\var}[1]{{\ensuremath{\text{\textit{#1}}}}}

\newenvironment{code}[1]{\begin{tabbing}
\hspace*{6mm}\=\hspace{6mm}\=\hspace{6mm}\=\hspace{6mm}\=\hspace{6mm}\=\hspace{6mm}\=\hspace{6mm}\=\hspace{6mm}\=\hspace{6mm}\=\hspace{6mm}\=\hspace{6mm}\=\hspace{6mm}\=\hspace{6mm}\=
\kill {{\tt Algorithm} {#1}}\\ \{\+ \\}{\-\\ \} \end{tabbing}}

\newenvironment{codeseg}{\begin{tabbing}
\hspace*{1em}\=\hspace{1em}\=\hspace{1em}\=\hspace{1em}\=\hspace{1em}\=\hspace{1em}\=\hspace{1em}\=\hspace{1em}\=\hspace{1em}\=\hspace{1em}\=\hspace{1em}\=\hspace{1em}\=\hspace{1em}\=
\kill }{\end{tabbing}}

\newenvironment{mylist}{%
  \begin{list}{}{%
      \renewcommand{\makelabel}[1]{%
        \hspace{\parindent}{\em{##1}}\quad}%
      \setlength{\parsep}{0mm}%
      \setlength{\labelsep}{0mm}%
      \setlength{\leftmargin}{0mm}%
      \setlength{\listparindent}{\parindent}}%
}{\end{list}}%
\newenvironment{txtlist}{%
  \begin{list}{}{%
      \renewcommand{\makelabel}[1]{%
%       \hspace{\parindent}%
        {\em{##1}}\quad%
      }
%      \setlength{\parsep}{0mm}%
%      \setlength{\labelsep}{0mm}%
%      \setlength{\leftmargin}{0mm}%
%      \setlength{\listparindent}{\parindent}%
}%
}{\end{list}}%

%%%%%%%%%%%%%%%%%%%%%%%%%%%%%%%%%%%%%%%%%%%%%%%%%%%%%%%%%%%%%%%
%%%%%%%%%%%%%%%%%%%%%%%%%%%%%%%%%%%%%%%%%%%%%%%%%%%%%%%%%%%%%%%
%%%%%%%%%%%%%%%%%%%%%%%%%%%%%%%%%%%%%%%%%%%%%%%%%%%%%%%%%%%%%%%
%%%%%%%%%%%%%%%%%%%%%%%%%%%%%%%%%%%%%%%%%%%%%%%%%%%%%%%%%%%%%%%

%\newcommand{\eqm}{\equiv}
\newcommand{\eqm}{=}
%% \renewcommand{\subset}{\subseteq}
%% \renewcommand{\supset}{\supseteq}
% \newcommand{\Pz}[1]{P_{#1}}
% \newcommand{\Pb}[1]{P^{\scriptscriptstyle 0/1}_{#1}}
% \newcommand{\Pm}[1]{P_{#1}}
% \newcommand{\Pse}[1]{P^{\text{se}}_{#1}}

%%% Local Variables: 
%%% mode: latex
%%% TeX-master: "nil"
%%% End: 
